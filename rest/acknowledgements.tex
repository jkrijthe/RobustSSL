\chapter{Acknowledgements}
Science is not -- metaphorically, although sometimes physically -- carried out in a vacuum. There is an intellectual basis that we build on: ideas, books, papers, code, a community. But there is also the personal support without which a four year research project would be a great deal less enjoyable. I am grateful to all the people who have helped me carry out the research that resulted in this thesis, who I will try to mention here.

As for the intellectual basis to fall back on, I am thankful for the wonderful world of statistics books and blogs for helping shape my statistical thinking. In particular: Andrew Gelman, Larry Wasserman, James O. Berger, Judea Pearl, Hadley Wickham and Edward Tufte. In the wonderful world of R programming, I am grateful to the people whose code I built on and all the volunteers responsible for helping me get my packages on CRAN. %I apologize for all the mistakes I made in the process.

The research in this thesis would not have been possible without Prof. Joost Kok and Prof. Eline Slagboom. Joost and Eline, thank you for giving me the opportunity that allowed me to explore these, at times esoteric, topics, while always keeping me focused on the overall goal of a finished thesis. I would like to thank Mark Kroon and Jonathan Vis, my COMMIT companions, and Jeroen Laros, for making the project meetings enjoyable and useful. I would also like to thank Nadine Mascini and Ron Heeren for their hospitality during my time at the AMOLF. And of course thanks to everybody at MolEpi, who put up with me during my times in Leiden, in particular Marian Beekman, Erik van den Akker, Joris Deelen and at earlier times Eric-Wubbo Lameijer and Kai Ye.

I thank everyone in the Pattern Recognition and Bioinformatics group in Delft who I had the pleasure to interact with over the years. Prof. Marcel Reinders, for his hospitality, by allowing me to be part of the PRB group. Erdogan, Sjoerd and Ahmed for showing me what bioinformatics is all about and welcoming my small contributions. Laura, Ekin, Hayley, Marieke, Lu, Hamdi, Yazhou, Alex, Taygun, Julian, Laurens, Jan, Emile, Wenjie, Yuanhao, Cuong, Yan and Bob for the coffeetalks, discussions and general merrymaking. David for all the observant questions. Veronika for being my trusted office mate and for all the fun activities outside the office. Wouter, for all the scientific discussions, friendship and your never ending enthusiasm for anything we come up with.

Marco, some of my fondest memories from the past four years are of our times together in front of the whiteboard, discussing machine learning, science, academia and the important things in life. The ideas presented in this thesis are as much yours as they are mine. To hark back to your own disseration: if only I had taken the time to write down some more of them... For those I did write down, I hope I have done them some justice. Thank you for being a great mentor, for guiding me to unknowns and standing by as a supporter and a friend in times of need.
\vspace{3mm}

\begin{otherlanguage}{dutch}
Voor ik eindig wil ik waardering uitspreken naar mijn ouders voor hun nooit aflatende steun, ook wanneer mijn interesses misschien moeilijk te begrijpen zijn. Ali en Arie, voor de boterhammen met hagelslag. Kees en Margriet, voor de steun en de warmte van Terschelling. Arne, Roos en Adrian, voor de vriendschap door de jaren heen ondanks alle veranderingen en afstand. En mijn broers, Jelmer en Bouwe, voor de voorbeelden die jullie altijd zijn geweest waarnaar ik kan streven.

Ten slotte, mijn Imzadi, Anne-Lotte. Alles wat ik nodig had om dit proefschrift te kunnen schrijven, leerde ik niet voordat ik jou leerde kennen. Met jou waag ik graag de volgende sprong in het ongewisse: zonder pretentie, maar vol vertrouwen.
\end{otherlanguage}
